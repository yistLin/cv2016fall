\documentclass[14pt,a4paper]{extarticle}

\usepackage[margin=2.54cm]{geometry}
\usepackage[slantfont,boldfont]{xeCJK}
\usepackage{listings}
\usepackage{float}
\usepackage{caption}
\usepackage{lastpage}
\usepackage{graphicx}
\graphicspath{ {../images/} }

\usepackage{fancyhdr}
\pagestyle{fancy}
\fancyhf{}
\lhead{\small{\jobname.pdf}}
\rhead{\small{Computer Vision 2016 Fall}}
\cfoot{\small{\thepage\ of \pageref{LastPage}}}

\usepackage{color}
\usepackage{listings}
\definecolor{codeyellow}{rgb}{0.6,0.4,0.1}
\definecolor{codeblue}{rgb}{0.1,0.2,0.6}
\definecolor{darkgray}{rgb}{0.3,0.3,0.3}
\lstset{
	language=Python,
	basicstyle=\footnotesize,
	keywordstyle=\color{codeblue},
	commentstyle=\color{darkgray},
	stringstyle=\color{codeyellow},
	numbers=left,
	numberstyle=\footnotesize,
	stepnumber=1,
	numbersep=5pt,
	backgroundcolor=\color{white},
	showspaces=false,
	showstringspaces=false,
	showtabs=false,
	frame=single,
	tabsize=2,
	captionpos=b,
	breaklines=true,
	breakatwhitespace=false,
	escapeinside={\%*}{*)}
}

\setCJKmainfont{PingFang TC}

\title{Computer Vision\\Homework 5 Report}
\author{林義聖\\B03902048}
\date{\today}

\begin{document}

\maketitle
\thispagestyle{fancy}

\section{Introduction}

In this homework assignment, we're going to do \textit{Mathematical Morphology --- Gray Scaled Morphology} and try to implement four algorithm: erosion, dilation, opening, and closing.

I use \textit{Python} as my programming language and \textit{Pillow} as my image library. It is a fork of \textit{PIL}, which is the original image library of \textit{Python}. And I use \textit{Pillow} for reading input image, transfering image data into \textit{List} of \textit{Python}, and saving the processed data as a gray-scaled image.

\begin{figure}[H]
\centering
\includegraphics[scale=0.5]{lena.bmp}
\caption{original lena.bmp}
\label{fig:lena.bmp}
\end{figure}

\section{Program Structure}

There's only one program in my submission, and I use program parameters to decide which algorithm we want to apply on the input image. All the things done in my program is:
\begin{enumerate}
	\item Import required library
	\item Define kernel
	\item Open input image
	\item Decide which algorithm to apply to and process the image data
	\item Create a gray-scale image and put the result into image
	\item Save the output image
\end{enumerate}

\begin{lstlisting}
import sys
from PIL import Image

# get system arguments
...

# define the 3-5-5-5-3 kernel
kernel = [\
    (-1,-2),(0,-2),(1,-2),\
    (-2,-1),(-1,-1),(0,-1),(1,-1),(2,-1),\
    (-2,0),(-1,0),(0,0),(1,0),(2,0),\
    (-2,1),(-1,1),(0,1),(1,1),(2,1),\
    (-1,2),(0,2),(1,2)\
]

# the required 4 algorithms are implemented as functions
...

# main function
def main():
    # get input image
    try:
        in_img = Image.open(infilename)
    except Exception as e:
        print('Error:', str(e))
        exit(1)
    ...

    # transfer image data into Python list
    data_seq = list(in_img.getdata())

    # handle which algorithm we want to apply on image
    if option == 'erosioin':
    	...

    # save output image
    out_img = Image.new('L', in_img.size)
    out_img.putdata(out_data)
    out_img.save(outfilename, 'bmp')

if __name__ == '__main__':
    main()
\end{lstlisting}

\section{Dilation $f \oplus k$}

The definition of \textit{Gray Scale Dilation} is,
\[ f \oplus k = T\{U[f] \oplus U[k] \} \]
When we are implementing this algorithm, we use this definition,
\[ (f \oplus k)(x) = \texttt{max}\{ f(x-z) + k(z)\ |\ z \in K, x-z \in F \} \]
Therefore, to do dilation, I use following steps:
\begin{enumerate}
	\item Loop through every pixels
	\item For each pixel, get value of translated pixels and put into a \textit{list}
	\item Find the maximum value in the \textit{list}
	\item Assign that value to the current pixel
\end{enumerate}

\begin{lstlisting}
def dilation(data, hei, wid):
    global kernel
    result = [0] * len(data)
    for y in range(hei):
        for x in range(wid):
            translated = []
            for x_k, y_k in kernel:
                x_t = x + x_k
                y_t = y + y_k
                if 0 <= x_t < wid and 0 <= y_t < hei:
                    translated.append(data[y_t * wid + x_t])
                else:
                    pass
            result[y * wid + x] = max(translated)
    return result
\end{lstlisting}

\begin{figure}[H]
\centering
\includegraphics[scale=0.6]{lena-dilated.bmp}
\caption{dilated lena.bmp}
\label{fig:lena-dilated.bmp}
\end{figure}

\section{Erosion $f \ominus k$}

The definition of \textit{Gray Scale Erosion} is,
\[ f \ominus k = T\{U[f] \ominus U[k] \} \]
When we are implementing this algorithm, we use this definition,
\[ (f \ominus k)(x) = \texttt{min}\{ f(x+z) - k(z) \} \]
Therefore, to do erosion, I use following steps:
\begin{enumerate}
	\item Loop through every pixels
	\item For each pixel, get value of translated pixels and put into a \textit{list}
	\item Find the minimum value in the \textit{list}
	\item Assign that value to the current pixel
\end{enumerate}

\begin{lstlisting}
def erosion(data, hei, wid):
    global kernel
    result = [0] * len(data)
    for y in range(hei):
        for x in range(wid):
            translated = []
            for x_k, y_k in kernel:
                x_t = x + x_k
                y_t = y + y_k
                if 0 <= x_t < wid and 0 <= y_t < hei:
                    translated.append(data[y_t * wid + x_t])
                else:
                    pass
            result[y * wid + x] = min(translated)
    return result
\end{lstlisting}

\begin{figure}[H]
\centering
\includegraphics[scale=0.6]{lena-eroded.bmp}
\caption{eroded lena.bmp}
\label{fig:lena-eroded.bmp}
\end{figure}

\section{Opening $f \circ k$}

The \textit{opening} on $f$ by $k$ is denoted by $f \circ k$. And it is defined by \textit{dilation} and \textit{erosion}, that is,
\[ f \circ k = (f \ominus k) \oplus k \]
Hence, to do \textit{opening}, I just apply \textit{erosion} on image first, and then apply \textit{dilation} on the result of previous step.

\begin{lstlisting}
def opening(data, hei, wid):
    return dilation( erosion(data, hei, wid), hei, wid )
\end{lstlisting}

\begin{figure}[H]
\centering
\includegraphics[scale=0.6]{lena-opening.bmp}
\caption{lena.bmp after opening}
\label{fig:lena-opening.bmp}
\end{figure}

\section{Closing $f \bullet k$}

The \textit{closing} on $f$ by $k$ is denoted by $f \bullet k$. And it is defined by \textit{dilation} and \textit{erosion}, that is,
\[ f \bullet k = (f \oplus k) \ominus k \]
Hence, to do \textit{closing}, I just apply \textit{dilation} on image first, and then apply \textit{erosion} on the result of previous step.

\begin{lstlisting}
def closing(data, hei, wid):
    return erosion( dilation(data, hei, wid), hei, wid )
\end{lstlisting}

\begin{figure}[H]
\centering
\includegraphics[scale=0.6]{lena-closing.bmp}
\caption{lena.bmp after closing}
\label{fig:lena-closing.bmp}
\end{figure}

\section{How to Use}

There's only one executable program in my submission, and its name is \textit{gs-morphology.py}. To run this program, just type this command ''\texttt{./gs-morphology.py [option] [input image] [output image]}''. And the \texttt{[option]} could be: dilation, erosion, opening, closing.

\end{document}
