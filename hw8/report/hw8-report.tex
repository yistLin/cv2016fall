\documentclass[14pt,a4paper]{extarticle}

\usepackage[margin=2.54cm]{geometry}
\usepackage[slantfont,boldfont]{xeCJK}
\usepackage{float}
\usepackage{caption}
\usepackage{lastpage}
\usepackage{graphicx}
\usepackage{amsmath}
\usepackage{subfig}
\graphicspath{ {../} }

\usepackage{fancyhdr}
\pagestyle{fancy}
\fancyhf{}
\lhead{\small{\jobname.pdf}}
\rhead{\small{Computer Vision 2016 Fall}}
\cfoot{\small{\thepage\ of \pageref{LastPage}}}

\usepackage{color}
\usepackage{listings}
\definecolor{codeyellow}{rgb}{0.6,0.4,0.1}
\definecolor{codeblue}{rgb}{0.1,0.2,0.6}
\definecolor{darkgray}{rgb}{0.3,0.3,0.3}
\lstset{
	language=Python,
	basicstyle=\footnotesize,
	keywordstyle=\color{codeblue},
	commentstyle=\color{darkgray},
	stringstyle=\color{codeyellow},
	numbers=left,
	numberstyle=\footnotesize,
	stepnumber=1,
	numbersep=5pt,
	backgroundcolor=\color{white},
	showspaces=false,
	showstringspaces=false,
	showtabs=false,
	frame=single,
	tabsize=2,
	captionpos=b,
	breaklines=true,
	breakatwhitespace=false,
	escapeinside={\%*}{*)}
}

\setCJKmainfont{Kaiti TC}
\newfontinstance{\optima}{Optima}

\title{{\optima Computer Vision\\Homework 8 Report}}
\author{林義聖\\B03902048}
\date{\today}

\begin{document}

\maketitle
\thispagestyle{fancy}

\section{Introduction}

In this homework assignment, we're going to generate some noisy image and apply filters to remove the noise. I use \textit{Python} as my programming language and \textit{Pillow} as my image library. It is a fork of \textit{PIL}, which is the original image library of \textit{Python}. And I use \textit{Pillow} for reading input image, and transfering image data into \textit{List} of \textit{Python}.

\begin{figure}[H]
\centering
\includegraphics[scale=0.5]{lena.bmp}
\caption{lena.bmp as Benchmark Image}
\label{fig:lena.bmp}
\end{figure}

\section{Generate Noisy Image}

\subsection{Additive White Gaussian Noise}

To generate additive white gaussian noise on given image, we use the following function and let $amplitude = 10 \text{ or } 30$,
\[ I(nim,i,j) = I(im,i,j) + amplitude \cdot N(0,1) \]

\begin{lstlisting}
def additive_white(data, amplitude):
	ret = []
	for pix in data:
		ret.append(pix + amplitude * random.gauss(0, 1))
	return ret
\end{lstlisting}

\begin{figure}[H]
\centering
	\subfloat[original picture] {
    \includegraphics[width=0.45\linewidth]{lena.bmp}
    \label{fig:lena.bmp}
	}
	\hfill
	\subfloat[Amplitude = 10] {
    \includegraphics[width=0.45\linewidth]{images/additive-white-10-lena.bmp}
    \label{fig:additive-white-10-lena.bmp}
	}
\caption{Additive white gaussian noise (amplitude = 10)}
\end{figure}

\begin{figure}[H]
\centering
	\subfloat[original picture] {
    \includegraphics[width=0.45\linewidth]{lena.bmp}
    \label{fig:lena.bmp}
	}
	\hfill
	\subfloat[Amplitude = 30] {
    \includegraphics[width=0.45\linewidth]{images/additive-white-30-lena.bmp}
    \label{fig:additive-white-30-lena.bmp}
	}
\caption{Additive white gaussian noise (amplitude = 30)}
\end{figure}

\subsection{Salt-and-Pepper Noise}

To generate salt-and-pepper noise on given image, we use the following function and let $threshold = 0.1 \text{ or } 0.05$,
\begin{equation*}
I(nim,i,j) =
\left\{
	\begin{aligned}
	& 0        & &\text{if uniform(0,1)} < \frac{\text{threshold}}{2}\\
	& 255      & &\text{if uniform(0,1)} > 1 - \frac{\text{threshold}}{2}\\
	& I(im,i,j)& &\text{otherwise}
	\end{aligned}
\right.
\end{equation*}

\begin{lstlisting}
def salt_and_pepper(data, threshold):
	ret = []
	threshold = threshold / 2
	for pix in data:
		uni = random.uniform(0, 1)
		if uni < threshold:
			ret.append(0)
		elif uni > (1 - threshold):
			ret.append(255)
		else:
			ret.append(pix)
	return ret
\end{lstlisting}

\begin{figure}[H]
\centering
	\subfloat[original picture] {
    \includegraphics[width=0.45\linewidth]{lena.bmp}
    \label{fig:lena.bmp}
	}
	\hfill
	\subfloat[Threshold = 0.05] {
    \includegraphics[width=0.45\linewidth]{images/salt-and-pepper-0.05-lena.bmp}
    \label{fig:salt-and-pepper-0.05-lena.bmp}
	}
\caption{Salt-and-pepper noise (threshold = 0.05)}
\end{figure}

\begin{figure}[H]
\centering
	\subfloat[original picture] {
    \includegraphics[width=0.45\linewidth]{lena.bmp}
    \label{fig:lena.bmp}
	}
	\hfill
	\subfloat[Threshold = 0.1] {
    \includegraphics[width=0.45\linewidth]{images/salt-and-pepper-0.1-lena.bmp}
    \label{fig:salt-and-pepper-0.1-lena.bmp}
	}
\caption{Salt-and-pepper noise (threshold = 0.1)}
\end{figure}

\section{Noise Removal}

\subsection{Box Filter}

To use box filter, I just calculate the average of $3 \times 3$ or $5 \times 5$ neighbors of each pixel. And for border pixel, I just ignore the pixels outside the boundaries. The following is the $3 \times 3$ box filter.
\begin{lstlisting}
def box_filter3x3(data, hei, wid):
	ret = []
	for y in range(hei):
		for x in range(wid):
			s = 0
			t = 0
			for off_x, off_y in product(range(-1,2), range(-1,2)):
				kx, ky = x + off_x, y + off_y
				if 0 <= kx < wid and 0 <= ky < hei:
					s += data[ky * wid + kx]
					t += 1
			ret.append(s / t)
	return ret
\end{lstlisting}

\begin{figure}[H]
\centering
	\subfloat[additive white (amplitude = 10)] {
    \includegraphics[width=0.45\linewidth]{images/additive-white-10-lena.bmp}
    \label{fig:additive-white-10-lena.bmp}
	}
	\hfill
	\subfloat[box filter $3 \times 3$ applied] {
    \includegraphics[width=0.45\linewidth]{images/box-filter3x3-additive-white-10-lena.bmp}
    \label{fig:box-filter3x3-additive-white-10-lena.bmp}
	}
\caption{Apply box filter $3 \times 3$}
\end{figure}

\begin{figure}[H]
\centering
	\subfloat[additive white (amplitude = 10)] {
    \includegraphics[width=0.45\linewidth]{images/additive-white-10-lena.bmp}
    \label{fig:additive-white-10-lena.bmp}
	}
	\hfill
	\subfloat[box filter $5 \times 5$ applied] {
    \includegraphics[width=0.45\linewidth]{images/box-filter5x5-additive-white-10-lena.bmp}
    \label{fig:box-filter5x5-additive-white-10-lena.bmp}
	}
\caption{Apply box filter $5 \times 5$}
\end{figure}

\begin{figure}[H]
\centering
	\subfloat[additive white (amplitude = 30)] {
    \includegraphics[width=0.45\linewidth]{images/additive-white-30-lena.bmp}
    \label{fig:additive-white-30-lena.bmp}
	}
	\hfill
	\subfloat[box filter $3 \times 3$ applied] {
    \includegraphics[width=0.45\linewidth]{images/box-filter3x3-additive-white-30-lena.bmp}
    \label{fig:box-filter3x3-additive-white-30-lena.bmp}
	}
\caption{Apply box filter $3 \times 3$}
\end{figure}

\begin{figure}[H]
\centering
	\subfloat[additive white (amplitude = 30)] {
    \includegraphics[width=0.45\linewidth]{images/additive-white-30-lena.bmp}
    \label{fig:additive-white-30-lena.bmp}
	}
	\hfill
	\subfloat[box filter $5 \times 5$ applied] {
    \includegraphics[width=0.45\linewidth]{images/box-filter5x5-additive-white-30-lena.bmp}
    \label{fig:box-filter5x5-additive-white-30-lena.bmp}
	}
\caption{Apply box filter $5 \times 5$}
\end{figure}

\begin{figure}[H]
\centering
	\subfloat[salt-and-pepper (threshold = 0.05)] {
    \includegraphics[width=0.45\linewidth]{images/salt-and-pepper-0.05-lena.bmp}
    \label{fig:salt-and-pepper-0.05-lena.bmp}
	}
	\hfill
	\subfloat[box filter $3 \times 3$ applied] {
    \includegraphics[width=0.45\linewidth]{images/box-filter3x3-salt-and-pepper-0.05-lena.bmp}
    \label{fig:box-filter3x3-salt-and-pepper-0.05-lena.bmp}
	}
\caption{Apply box filter $3 \times 3$}
\end{figure}

\begin{figure}[H]
\centering
	\subfloat[salt-and-pepper (threshold = 0.05)] {
    \includegraphics[width=0.45\linewidth]{images/salt-and-pepper-0.05-lena.bmp}
    \label{fig:salt-and-pepper-0.05-lena.bmp}
	}
	\hfill
	\subfloat[box filter $5 \times 5$ applied] {
    \includegraphics[width=0.45\linewidth]{images/box-filter5x5-salt-and-pepper-0.05-lena.bmp}
    \label{fig:box-filter5x5-salt-and-pepper-0.05-lena.bmp}
	}
\caption{Apply box filter $5 \times 5$}
\end{figure}

\begin{figure}[H]
\centering
	\subfloat[salt-and-pepper (threshold = 0.1)] {
    \includegraphics[width=0.45\linewidth]{images/salt-and-pepper-0.1-lena.bmp}
    \label{fig:salt-and-pepper-0.1-lena.bmp}
	}
	\hfill
	\subfloat[box filter $3 \times 3$ applied] {
    \includegraphics[width=0.45\linewidth]{images/box-filter3x3-salt-and-pepper-0.1-lena.bmp}
    \label{fig:box-filter3x3-salt-and-pepper-0.1-lena.bmp}
	}
\caption{Apply box filter $3 \times 3$}
\end{figure}

\begin{figure}[H]
\centering
	\subfloat[salt-and-pepper (threshold = 0.1)] {
    \includegraphics[width=0.45\linewidth]{images/salt-and-pepper-0.1-lena.bmp}
    \label{fig:salt-and-pepper-0.1-lena.bmp}
	}
	\hfill
	\subfloat[box filter $5 \times 5$ applied] {
    \includegraphics[width=0.45\linewidth]{images/box-filter5x5-salt-and-pepper-0.1-lena.bmp}
    \label{fig:box-filter5x5-salt-and-pepper-0.1-lena.bmp}
	}
\caption{Apply box filter $5 \times 5$}
\end{figure}

\subsection{Median Filter}

To use median filter, I just find the median of $3 \times 3$ or $5 \times 5$ neighbors of each pixel. And for border pixel, I just ignore the pixels outside the boundaries. The following is the $3 \times 3$ median filter.
\begin{lstlisting}
def median_filter3x3(data, hei, wid):
	ret = []
	for y in range(hei):
		for x in range(wid):
			l = []
			for off_x, off_y in product(range(-1,2), range(-1,2)):
				kx, ky = x + off_x, y + off_y
				if 0 <= kx < wid and 0 <= ky < hei:
					l.append(data[ky * wid + kx])
			ret.append(median(l))
	return ret
\end{lstlisting}

\begin{figure}[H]
\centering
	\subfloat[additive white (amplitude = 10)] {
    \includegraphics[width=0.45\linewidth]{images/additive-white-10-lena.bmp}
    \label{fig:additive-white-10-lena.bmp}
	}
	\hfill
	\subfloat[median filter $3 \times 3$ applied] {
    \includegraphics[width=0.45\linewidth]{images/median-filter3x3-additive-white-10-lena.bmp}
    \label{fig:median-filter3x3-additive-white-10-lena.bmp}
	}
\caption{Apply median filter $3 \times 3$}
\end{figure}

\begin{figure}[H]
\centering
	\subfloat[additive white (amplitude = 10)] {
    \includegraphics[width=0.45\linewidth]{images/additive-white-10-lena.bmp}
    \label{fig:additive-white-10-lena.bmp}
	}
	\hfill
	\subfloat[median filter $5 \times 5$ applied] {
    \includegraphics[width=0.45\linewidth]{images/median-filter5x5-additive-white-10-lena.bmp}
    \label{fig:median-filter5x5-additive-white-10-lena.bmp}
	}
\caption{Apply median filter $5 \times 5$}
\end{figure}

\begin{figure}[H]
\centering
	\subfloat[additive white (amplitude = 30)] {
    \includegraphics[width=0.45\linewidth]{images/additive-white-30-lena.bmp}
    \label{fig:additive-white-30-lena.bmp}
	}
	\hfill
	\subfloat[median filter $3 \times 3$ applied] {
    \includegraphics[width=0.45\linewidth]{images/median-filter3x3-additive-white-30-lena.bmp}
    \label{fig:median-filter3x3-additive-white-30-lena.bmp}
	}
\caption{Apply median filter $3 \times 3$}
\end{figure}

\begin{figure}[H]
\centering
	\subfloat[additive white (amplitude = 30)] {
    \includegraphics[width=0.45\linewidth]{images/additive-white-30-lena.bmp}
    \label{fig:additive-white-30-lena.bmp}
	}
	\hfill
	\subfloat[median filter $5 \times 5$ applied] {
    \includegraphics[width=0.45\linewidth]{images/median-filter5x5-additive-white-30-lena.bmp}
    \label{fig:median-filter5x5-additive-white-30-lena.bmp}
	}
\caption{Apply median filter $5 \times 5$}
\end{figure}

\begin{figure}[H]
\centering
	\subfloat[salt-and-pepper (threshold = 0.05)] {
    \includegraphics[width=0.45\linewidth]{images/salt-and-pepper-0.05-lena.bmp}
    \label{fig:salt-and-pepper-0.05-lena.bmp}
	}
	\hfill
	\subfloat[median filter $3 \times 3$ applied] {
    \includegraphics[width=0.45\linewidth]{images/median-filter3x3-salt-and-pepper-0.05-lena.bmp}
    \label{fig:median-filter3x3-salt-and-pepper-0.05-lena.bmp}
	}
\caption{Apply median filter $3 \times 3$}
\end{figure}

\begin{figure}[H]
\centering
	\subfloat[salt-and-pepper (threshold = 0.05)] {
    \includegraphics[width=0.45\linewidth]{images/salt-and-pepper-0.05-lena.bmp}
    \label{fig:salt-and-pepper-0.05-lena.bmp}
	}
	\hfill
	\subfloat[median filter $5 \times 5$ applied] {
    \includegraphics[width=0.45\linewidth]{images/median-filter5x5-salt-and-pepper-0.05-lena.bmp}
    \label{fig:median-filter5x5-salt-and-pepper-0.05-lena.bmp}
	}
\caption{Apply median filter $5 \times 5$}
\end{figure}

\begin{figure}[H]
\centering
	\subfloat[salt-and-pepper (threshold = 0.1)] {
    \includegraphics[width=0.45\linewidth]{images/salt-and-pepper-0.1-lena.bmp}
    \label{fig:salt-and-pepper-0.1-lena.bmp}
	}
	\hfill
	\subfloat[median filter $3 \times 3$ applied] {
    \includegraphics[width=0.45\linewidth]{images/median-filter3x3-salt-and-pepper-0.1-lena.bmp}
    \label{fig:median-filter3x3-salt-and-pepper-0.1-lena.bmp}
	}
\caption{Apply median filter $3 \times 3$}
\end{figure}

\begin{figure}[H]
\centering
	\subfloat[salt-and-pepper (threshold = 0.1)] {
    \includegraphics[width=0.45\linewidth]{images/salt-and-pepper-0.1-lena.bmp}
    \label{fig:salt-and-pepper-0.1-lena.bmp}
	}
	\hfill
	\subfloat[median filter $5 \times 5$ applied] {
    \includegraphics[width=0.45\linewidth]{images/median-filter5x5-salt-and-pepper-0.1-lena.bmp}
    \label{fig:median-filter5x5-salt-and-pepper-0.1-lena.bmp}
	}
\caption{Apply median filter $5 \times 5$}
\end{figure}

\subsection{Opening-then-Closing}

To doing opening followed by closing, I just apply what I have done in the previous assignment.
\begin{lstlisting}
def opening(data, hei, wid):
	return dilation( erosion(data, hei, wid), hei, wid )

def closing(data, hei, wid):
	return erosion( dilation(data, hei, wid), hei, wid )

...
data = opening(pixellist, hei, wid)
data = closing(data, hei, wid)
\end{lstlisting}

\begin{figure}[H]
\centering
	\subfloat[additive white (amplitude = 10)] {
    \includegraphics[width=0.45\linewidth]{images/additive-white-10-lena.bmp}
    \label{fig:additive-white-10-lena.bmp}
	}
	\hfill
	\subfloat[opening-then-closing applied] {
    \includegraphics[width=0.45\linewidth]{images/opening-closing-additive-white-10-lena.bmp}
    \label{fig:opening-closing-additive-white-10-lena.bmp}
	}
\caption{Apply opening-then-closing}
\end{figure}

\begin{figure}[H]
\centering
	\subfloat[additive white (amplitude = 30)] {
    \includegraphics[width=0.45\linewidth]{images/additive-white-30-lena.bmp}
    \label{fig:additive-white-30-lena.bmp}
	}
	\hfill
	\subfloat[opening-then-closing applied] {
    \includegraphics[width=0.45\linewidth]{images/opening-closing-additive-white-30-lena.bmp}
    \label{fig:opening-closing-additive-white-30-lena.bmp}
	}
\caption{Apply opening-then-closing}
\end{figure}

\begin{figure}[H]
\centering
	\subfloat[salt-and-pepper (threshold = 0.05)] {
    \includegraphics[width=0.45\linewidth]{images/salt-and-pepper-0.05-lena.bmp}
    \label{fig:salt-and-pepper-0.05-lena.bmp}
	}
	\hfill
	\subfloat[opening-then-closing applied] {
    \includegraphics[width=0.45\linewidth]{images/opening-closing-salt-and-pepper-0.05-lena.bmp}
    \label{fig:opening-closing-salt-and-pepper-0.05-lena.bmp}
	}
\caption{Apply opening-then-closing}
\end{figure}

\begin{figure}[H]
\centering
	\subfloat[salt-and-pepper (threshold = 0.1)] {
    \includegraphics[width=0.45\linewidth]{images/salt-and-pepper-0.1-lena.bmp}
    \label{fig:salt-and-pepper-0.1-lena.bmp}
	}
	\hfill
	\subfloat[opening-then-closing applied] {
    \includegraphics[width=0.45\linewidth]{images/opening-closing-salt-and-pepper-0.1-lena.bmp}
    \label{fig:opening-closing-salt-and-pepper-0.1-lena.bmp}
	}
\caption{Apply opening-then-closing}
\end{figure}

\subsection{Closing-then-Opening}

To doing closing followed by opening, I just apply what I have done in the previous assignment.
\begin{lstlisting}
def opening(data, hei, wid):
	return dilation( erosion(data, hei, wid), hei, wid )

def closing(data, hei, wid):
	return erosion( dilation(data, hei, wid), hei, wid )

...
data = closing(pixellist, hei, wid)
data = opening(data, hei, wid)
\end{lstlisting}

\begin{figure}[H]
\centering
	\subfloat[additive white (amplitude = 10)] {
    \includegraphics[width=0.45\linewidth]{images/additive-white-10-lena.bmp}
    \label{fig:additive-white-10-lena.bmp}
	}
	\hfill
	\subfloat[closing-then-opening applied] {
    \includegraphics[width=0.45\linewidth]{images/closing-opening-additive-white-10-lena.bmp}
    \label{fig:closing-opening-additive-white-10-lena.bmp}
	}
\caption{Apply closing-then-opening}
\end{figure}

\begin{figure}[H]
\centering
	\subfloat[additive white (amplitude = 30)] {
    \includegraphics[width=0.45\linewidth]{images/additive-white-30-lena.bmp}
    \label{fig:additive-white-30-lena.bmp}
	}
	\hfill
	\subfloat[closing-then-opening applied] {
    \includegraphics[width=0.45\linewidth]{images/closing-opening-additive-white-30-lena.bmp}
    \label{fig:closing-opening-additive-white-30-lena.bmp}
	}
\caption{Apply closing-then-opening}
\end{figure}

\begin{figure}[H]
\centering
	\subfloat[salt-and-pepper (threshold = 0.05)] {
    \includegraphics[width=0.45\linewidth]{images/salt-and-pepper-0.05-lena.bmp}
    \label{fig:salt-and-pepper-0.05-lena.bmp}
	}
	\hfill
	\subfloat[closing-then-opening applied] {
    \includegraphics[width=0.45\linewidth]{images/closing-opening-salt-and-pepper-0.05-lena.bmp}
    \label{fig:closing-opening-salt-and-pepper-0.05-lena.bmp}
	}
\caption{Apply closing-then-opening}
\end{figure}

\begin{figure}[H]
\centering
	\subfloat[salt-and-pepper (threshold = 0.1)] {
    \includegraphics[width=0.45\linewidth]{images/salt-and-pepper-0.1-lena.bmp}
    \label{fig:salt-and-pepper-0.1-lena.bmp}
	}
	\hfill
	\subfloat[closing-then-opening applied] {
    \includegraphics[width=0.45\linewidth]{images/closing-opening-salt-and-pepper-0.1-lena.bmp}
    \label{fig:closing-opening-salt-and-pepper-0.1-lena.bmp}
	}
\caption{Apply closing-then-opening}
\end{figure}

\subsection{Signal-to-ratio}

\begin{lstlisting}
# ori_pixellist is the data list from original image
# proc_pixellist is the data list from processed image

# calculate mu, mu_n
mu = sum(ori_pixellist) / len(ori_pixellist)
mu_n = sum([ (proc_pixellist[i] - ori_pixellist[i]) for i in range(len(proc_pixellist)) ]) / len(proc_pixellist)

# calculate VS
vs = sum( [(p - mu)**2 for p in ori_pixellist] ) / len(ori_pixellist)
vn = sum( [(proc_pixellist[i] - ori_pixellist[i] - mu_n)**2 for i in range(len(proc_pixellist))] ) / len(proc_pixellist)

# calculate SNR
snr = 10 * log10(vs / vn)
\end{lstlisting}

And the result of the calculation is,
\lstset{
	language=,
	basicstyle=\footnotesize,
	backgroundcolor=\color{white},
	numbers=none,
	showspaces=false,
	showstringspaces=false,
	showtabs=false,
	frame=single,
	tabsize=2,
	captionpos=b,
	breaklines=true,
	breakatwhitespace=false,
	escapeinside={\%*}{*)}
}
\lstinputlisting{../SNR.txt}

\section{How to Use}

There are some executable program in my submission, they are
\begin{itemize}
	\item \textit{generate\_noise.py}
	\item \textit{noise\_removal.py}
	\item \textit{calc\_SNR.py}
\end{itemize}
To use \textit{generate\_noise.py}, you need to type ''\texttt{./generate\_noise.py [input image]}''. To use \textit{noise\_removal.py}, you need to type ''\texttt{./noise\_removal.py --filter=[box|median] [input image]}''. To use \textit{calc\_SNR.py}, you need to type \texttt{./calc\_SNR.py [original image] [processed image]}.

\end{document}
