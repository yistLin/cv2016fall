\documentclass[14pt,a4paper]{extarticle}

\usepackage[margin=2.54cm]{geometry}
\usepackage[slantfont,boldfont]{xeCJK}
\usepackage[utf8]{inputenc}
\usepackage{listings}
\usepackage{float}
\usepackage{color}
\usepackage{xcolor}
\usepackage{caption}
\usepackage{courier}
\usepackage{subfig}
\usepackage{graphicx}
\graphicspath{ {./} }

\definecolor{codegreen}{rgb}{0,0.6,0}
\definecolor{codepurple}{rgb}{0.58,0,0.82}
\definecolor{backcolour}{rgb}{0.95,0.95,0.92}
\definecolor{darkgray}{rgb}{0.3,0.3,0.3}

\newcommand{\inlinecode}[1]{\colorbox{backcolour}{\color{darkgray}#1}}

\DeclareCaptionFont{white}{\color{white}}
\DeclareCaptionFormat{listing}{\colorbox{gray}{\parbox{\textwidth}{#1#2#3}}}
\captionsetup[lstlisting]{format=listing,labelfont=white,textfont=white}
\lstset{
	backgroundcolor=\color{backcolour},
	commentstyle=\color{codegreen},
    keywordstyle=\color{violet},
    stringstyle=\color{codepurple},
	language=Python,
	numbers=left,
	numberstyle=\small\color{lightgray},
	columns=fullflexible,
	showstringspaces=false,
	breaklines=true
}

\setCJKmainfont{PingFang TC}

\title{Computer Vision\\Homework 4 Report}
\author{林義聖\\B03902048}
\date{\today}

\begin{document}

\maketitle

\section*{Introduction}

I use \textit{Python} as my programming language and \textit{Pillow} as my Image Library.

\begin{figure}[H]
\centering
\includegraphics[scale=0.5]{lena.bmp}
\caption{original lena.bmp}
\label{fig:lena.bmp}
\end{figure}

\section*{Erosion}

\begin{lstlisting}[caption=Erosion]
mask_2d = [\
    (-1,-2),(0,-2),(1,-2),\
    (-2,-1),(-1,-1),(0,-1),(1,-1),(2,-1),\
    (-2,0),(-1,0),(0,0),(1,0),(2,0),\
    (-2,1),(-1,1),(0,1),(1,1),(2,1),\
    (-1,2),(0,2),(1,2)\
]

for y in range(height):
    for x in range(width):
        if data_seq[y * width + x] == WHITE:
            for m in mask_2d:
                p = (x + m[0], y + m[1])
                if p[0] < 0 or p[0] >= width or p[1] < 0 or p[1] >= height:
                    break
                elif data_seq[p[1] * width + p[0]] != WHITE:
                    break
            else:
                out_data[y * width + x] = WHITE
\end{lstlisting}

\begin{figure}[H]
\centering
\includegraphics[scale=0.6]{lena-eroded.bmp}
\caption{eroded lena.bmp}
\label{fig:lena-eroded.bmp}
\end{figure}

\section*{Dilation}

When I am using dilation on the image, I use the same mask defined in previous part.

\begin{lstlisting}[caption=Dilation]
for y in range(height):
    for x in range(width):
        if data_seq[y * width + x] == WHITE:
            for m in mask_2d:
                p = (x + m[0], y + m[1])
                if 0 <= p[0] < width and 0 <= p[1] < height:
                    out_data[p[1] * width + p[0]] = WHITE
\end{lstlisting}

\begin{figure}[H]
\centering
\includegraphics[scale=0.6]{lena-dilated.bmp}
\caption{dilated lena.bmp}
\label{fig:lena-dilated.bmp}
\end{figure}

\section*{Opening}

For doing \textit{opening} on the image, I use erosion followed by dilation with the same kernel.

\begin{figure}[H]
\centering
\includegraphics[scale=0.6]{lena-opening.bmp}
\caption{lena.bmp after opening}
\label{fig:lena-opening.bmp}
\end{figure}

\section*{Closing}

For doing \textit{closing} on the image, I use dilation followed by erosion with the same kernel.

\begin{figure}[H]
\centering
\includegraphics[scale=0.6]{lena-closing.bmp}
\caption{lena.bmp after closing}
\label{fig:lena-closing.bmp}
\end{figure}

\section*{Hit-and-Miss}

I draw white points on a black picture to represent the detected upper-right corners on lena.bmp.

\begin{lstlisting}[caption=Hit-and-Miss]
mask_pos = [\
    (-1,-1),(0,-1),(1,-1),\
    (-1,0),(0,0),(1,0),\
    (-1,1),(0,1),(1,1)]

mask = [\
    0,-1,-1,\
    1,1,-1,\
    0,1,0]

for y in range(height):
    for x in range(width):
        if data_seq[y*width+x] == WHITE:
            for i in range(len(mask_pos)):
                p = (x + mask_pos[i][0], y + mask_pos[i][1])
                if 0 <= p[0] < width and 0 <= p[1] < height:
                    if mask[i] == 1 and data_seq[p[1]*width+p[0]] == WHITE:
                        continue
                    elif mask[i] == -1 and data_seq[p[1]*width+p[0]] == BLACK:
                        continue
                    elif mask[i] == 0:
                        continue
                    else:
                        break
            else:
                draw.point((x,y), fill=255)
\end{lstlisting}

\begin{figure}[H]
\centering
\includegraphics[scale=0.6]{lena-hit.bmp}
\caption{lena.bmp after hit-and-miss}
\label{fig:lena-hit.bmp}
\end{figure}

\section*{How to Use}
There are 5 programs,
\begin{enumerate}
    \item \textit{erosion.py}
    \item \textit{dilation.py}
    \item \textit{opening.py}
    \item \textit{closing.py}
    \item \textit{hit-and-miss.py}
\end{enumerate}
You need to use binarized picture as input and enter commands in this format: "\textit{program [input image name] [output image name]}" to use it. For example, \inlinecode{./erosion.py lena-binarized.bmp output.bmp}.

\end{document}
