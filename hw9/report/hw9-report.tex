\documentclass[14pt,a4paper]{extarticle}

\usepackage[margin=2.54cm]{geometry}
\usepackage[slantfont,boldfont]{xeCJK}
\usepackage{float}
\usepackage{caption}
\usepackage{lastpage}
\usepackage{graphicx}
\usepackage{amsmath}
\usepackage{subfig}
\graphicspath{ {../} }

\usepackage{fancyhdr}
\pagestyle{fancy}
\fancyhf{}
\lhead{\small{\jobname.pdf}}
\rhead{\small{Computer Vision 2016 Fall}}
\cfoot{\small{\thepage\ of \pageref{LastPage}}}

\usepackage{color}
\usepackage{listings}
\definecolor{codeyellow}{rgb}{0.6,0.4,0.1}
\definecolor{codeblue}{rgb}{0.1,0.2,0.6}
\definecolor{darkgray}{rgb}{0.3,0.3,0.3}
\lstset{
	language=Python,
	basicstyle=\footnotesize,
	keywordstyle=\color{codeblue},
	commentstyle=\color{darkgray},
	stringstyle=\color{codeyellow},
	numbers=left,
	numberstyle=\footnotesize,
	stepnumber=1,
	numbersep=5pt,
	backgroundcolor=\color{white},
	showspaces=false,
	showstringspaces=false,
	showtabs=false,
	frame=single,
	tabsize=2,
	captionpos=b,
	breaklines=true,
	breakatwhitespace=false,
	escapeinside={\%*}{*)}
}

\setCJKmainfont{Kaiti TC}
\newfontinstance{\optima}{Optima}

\title{{\optima Computer Vision\\Homework 9 Report}}
\author{林義聖\\B03902048}
\date{\today}

\begin{document}

\maketitle
\thispagestyle{fancy}

\section{Introduction}

In this homework assignment, we're going to do \textbf{General Edge Detection}. I use \textit{Python} as my programming language and \textit{Pillow} as my image library. It is a fork of \textit{PIL}, which is the original image library of \textit{Python}. And I use \textit{Pillow} for reading input image, and transfering image data into a \textit{List} of \textit{Python}.

\begin{figure}[H]
\centering
\includegraphics[scale=0.5]{lena.bmp}
\caption{lena.bmp as Benchmark Image}
\label{fig:lena.bmp}
\end{figure}

\section{General Edge Detection}

To do edge detection, I apply each kernel and calculate the gradient magnitude $g$ for each operator. When the kernel reach the boundary of the image, I use a function to flip the inward pixels to the outward.

\begin{lstlisting}
def flip(p, boundary):
	if p < 0:
		return -p -1
	elif p >= boundary:
		return 2 * boundary - p - 1
	else:
		return p
\end{lstlisting}

\subsection{Robert's Operator}

Gradient magnitude of this operator is $g = \sqrt{r_1^2 + r_2^2}$.

\begin{lstlisting}
def roberts_operator(data, hei, wid):
	r1 = [\
		( 0, 0,-1),( 1, 0, 0),\
		( 0, 1, 0),( 1, 1, 1)]
	r2 = [\
		( 0, 0, 0),( 1, 0,-1),\
		( 0, 1, 1),( 1, 1, 0)]
	ret = []
	for y in range(hei):
		for x in range(wid):
			r1val = 0
			for rx, ry, val in r1:
				px = flip(x + rx, wid)
				py = flip(y + ry, hei)
				r1val += data[py * wid + px] * val
			r2val = 0
			for rx, ry, val in r2:
				px = flip(x + rx, wid)
				py = flip(y + ry, hei)
				r2val += data[py * wid + px] * val
			val = sqrt(r1val ** 2 + r2val ** 2)
			ret.append(0 if val >= 12 else 255)
	return ret
\end{lstlisting}

\begin{figure}[H]
\centering
\includegraphics[scale=0.5]{images/roberts-lena.bmp}
\caption{Robert's operator with threshold = 12}
\end{figure}

\subsection{Prewitt's Edge Detector}

The method to calculate gradient magnitude of this operator is the same as previous one, and the kernel of this operator is,
\begin{lstlisting}
p1 = [\
	(-1,-1,-1),( 0,-1,-1),( 1,-1,-1),\
	(-1, 0, 0),( 0, 0, 0),( 1, 0, 0),\
	(-1, 1, 1),( 0, 1, 1),( 1, 1, 1)]
p2 = [\
	(-1,-1,-1),( 0,-1, 0),( 1,-1, 1),\
	(-1, 0,-1),( 0, 0, 0),( 1, 0, 1),\
	(-1, 1,-1),( 0, 1, 0),( 1, 1, 1)]
\end{lstlisting}

\begin{figure}[H]
\centering
\includegraphics[scale=0.5]{images/prewitt-lena.bmp}
\caption{Prewitt's edge detector with threshold = 24}
\end{figure}

\subsection{Sobel's Edge Detector}

The method to calculate gradient magnitude of this operator is the same as previous one, and the kernel of this operator is,
\begin{lstlisting}
s1 = [\
	(-1,-1,-1),( 0,-1,-2),( 1,-1,-1),\
	(-1, 0, 0),( 0, 0, 0),( 1, 0, 0),\
	(-1, 1, 1),( 0, 1, 2),( 1, 1, 1)]
s2 = [\
	(-1,-1,-1),( 0,-1, 0),( 1,-1, 1),\
	(-1, 0,-2),( 0, 0, 0),( 1, 0, 2),\
	(-1, 1,-1),( 0, 1, 0),( 1, 1, 1)]
\end{lstlisting}

\begin{figure}[H]
\centering
\includegraphics[scale=0.5]{images/sobel-lena.bmp}
\caption{Sobel's edge detector with threshold = 38}
\end{figure}

\subsection{Frei and Chen's Gradient Operator}

The method to calculate gradient magnitude of this operator is the same as previous one, and the kernel of this operator is,
\begin{lstlisting}
f1 = [\
	(-1,-1,-1),( 0,-1,-sqrt(2)),( 1,-1,-1),\
	(-1, 0, 0),( 0, 0,       0),( 1, 0, 0),\
	(-1, 1, 1),( 0, 1, sqrt(2)),( 1, 1, 1)]
f2 = [\
	(-1,-1,      -1),( 0,-1, 0),( 1,-1,       1),\
	(-1, 0,-sqrt(2)),( 0, 0, 0),( 1, 0, sqrt(2)),\
	(-1, 1,      -1),( 0, 1, 0),( 1, 1,       1)]
\end{lstlisting}

\begin{figure}[H]
\centering
\includegraphics[scale=0.5]{images/frei_and_chen-lena.bmp}
\caption{Frei and Chen's gradient operator with threshold = 30}
\end{figure}

\subsection{Kirsch's Compass Operator}

Gradient magnitude of this operator is $g = \underset{n,n=0,\dots,7}{\max}k_n$.

\begin{lstlisting}
def kirsch_operator(data, hei, wid):
	k = [
		[\
		(-1,-1,-3),( 0,-1,-3),( 1,-1, 5),\
		(-1, 0,-3),( 0, 0, 0),( 1, 0, 5),\
		(-1, 1,-3),( 0, 1,-3),( 1, 1, 5)],\
		[\
		(-1,-1,-3),( 0,-1, 5),( 1,-1, 5),\
		(-1, 0,-3),( 0, 0, 0),( 1, 0, 5),\
		(-1, 1,-3),( 0, 1,-3),( 1, 1,-3)],\
		[\
		(-1,-1, 5),( 0,-1, 5),( 1,-1, 5),\
		(-1, 0,-3),( 0, 0, 0),( 1, 0,-3),\
		(-1, 1,-3),( 0, 1,-3),( 1, 1,-3)],\
		[\
		(-1,-1, 5),( 0,-1, 5),( 1,-1,-3),\
		(-1, 0, 5),( 0, 0, 0),( 1, 0,-3),\
		(-1, 1,-3),( 0, 1,-3),( 1, 1,-3)],\
		[\
		(-1,-1, 5),( 0,-1,-3),( 1,-1,-3),\
		(-1, 0, 5),( 0, 0, 0),( 1, 0,-3),\
		(-1, 1, 5),( 0, 1,-3),( 1, 1,-3)],\
		[\
		(-1,-1,-3),( 0,-1,-3),( 1,-1,-3),\
		(-1, 0, 5),( 0, 0, 0),( 1, 0,-3),\
		(-1, 1, 5),( 0, 1, 5),( 1, 1,-3)],\
		[\
		(-1,-1,-3),( 0,-1,-3),( 1,-1,-3),\
		(-1, 0,-3),( 0, 0, 0),( 1, 0,-3),\
		(-1, 1, 5),( 0, 1, 5),( 1, 1, 5)],\
		[\
		(-1,-1,-3),( 0,-1,-3),( 1,-1,-3),\
		(-1, 0,-3),( 0, 0, 0),( 1, 0, 5),\
		(-1, 1,-3),( 0, 1, 5),( 1, 1, 5)]\
	]
	ret = []
	for y in range(hei):
		for x in range(wid):
			maxval = -float('Inf')
			for kernel in k:
				tmpval = 0
				for rx, ry, val in kernel:
					px = flip(x + rx, wid)
					py = flip(y + ry, hei)
					tmpval += data[py * wid + px] * val
				maxval = tmpval if tmpval > maxval else maxval
			ret.append(0 if maxval >= 135 else 255)
	return ret
\end{lstlisting}

\begin{figure}[H]
\centering
\includegraphics[scale=0.5]{images/kirsch-lena.bmp}
\caption{Kirsch's compass operator with threshold = 135}
\end{figure}

\subsection{Robinson's Compass Operator}

The method to calculate gradient magnitude of this operator is the same as previous one, and the kernel of this operator is,
\begin{lstlisting}
k = [
	[\
	(-1,-1,-1),( 0,-1, 0),( 1,-1, 1),\
	(-1, 0,-2),( 0, 0, 0),( 1, 0, 2),\
	(-1, 1,-1),( 0, 1, 0),( 1, 1, 1)],\
	[\
	(-1,-1, 0),( 0,-1, 1),( 1,-1, 2),\
	(-1, 0,-1),( 0, 0, 0),( 1, 0, 1),\
	(-1, 1,-2),( 0, 1,-1),( 1, 1, 0)],\
	[\
	(-1,-1, 1),( 0,-1, 2),( 1,-1, 1),\
	(-1, 0, 0),( 0, 0, 0),( 1, 0, 0),\
	(-1, 1,-1),( 0, 1,-2),( 1, 1,-1)],\
	[\
	(-1,-1, 2),( 0,-1, 1),( 1,-1, 0),\
	(-1, 0, 1),( 0, 0, 0),( 1, 0,-1),\
	(-1, 1, 0),( 0, 1,-1),( 1, 1,-2)],\
	[\
	(-1,-1, 1),( 0,-1, 0),( 1,-1,-1),\
	(-1, 0, 2),( 0, 0, 0),( 1, 0,-2),\
	(-1, 1, 1),( 0, 1, 0),( 1, 1,-1)],\
	[\
	(-1,-1, 0),( 0,-1,-1),( 1,-1,-2),\
	(-1, 0, 1),( 0, 0, 0),( 1, 0,-1),\
	(-1, 1, 2),( 0, 1, 1),( 1, 1, 0)],\
	[\
	(-1,-1,-1),( 0,-1,-2),( 1,-1,-1),\
	(-1, 0, 0),( 0, 0, 0),( 1, 0, 0),\
	(-1, 1, 1),( 0, 1, 2),( 1, 1, 1)],\
	[\
	(-1,-1,-2),( 0,-1,-1),( 1,-1, 0),\
	(-1, 0,-1),( 0, 0, 0),( 1, 0, 1),\
	(-1, 1, 0),( 0, 1, 1),( 1, 1, 2)]\
]
\end{lstlisting}

\begin{figure}[H]
\centering
\includegraphics[scale=0.5]{images/robinson-lena.bmp}
\caption{Robinson's compass operator with threshold = 43}
\end{figure}

\subsection{Nevatia-Babu $5\times5$ Operator}

The method to calculate gradient magnitude of this operator is the same as previous one, and the kernel of this operator is,
\begin{lstlisting}
k = [
	[\
	(-2,-2, 100),(-1,-2, 100),( 0,-2, 100),( 1,-2, 100),( 2,-2, 100),\
	(-2,-1, 100),(-1,-1, 100),( 0,-1, 100),( 1,-1, 100),( 2,-1, 100),\
	(-2, 0,   0),(-1, 0,   0),( 0, 0,   0),( 1, 0,   0),( 2, 0,   0),\
	(-2, 1,-100),(-1, 1,-100),( 0, 1,-100),( 1, 1,-100),( 2, 1,-100),\
	(-2, 2,-100),(-1, 2,-100),( 0, 2,-100),( 1, 2,-100),( 2, 2,-100)],\
	[\
	(-2,-2, 100),(-1,-2, 100),( 0,-2, 100),( 1,-2, 100),( 2,-2, 100),\
	(-2,-1, 100),(-1,-1, 100),( 0,-1, 100),( 1,-1,  78),( 2,-1, -32),\
	(-2, 0, 100),(-1, 0,  92),( 0, 0,   0),( 1, 0, -92),( 2, 0,-100),\
	(-2, 1,  32),(-1, 1, -78),( 0, 1,-100),( 1, 1,-100),( 2, 1,-100),\
	(-2, 2,-100),(-1, 2,-100),( 0, 2,-100),( 1, 2,-100),( 2, 2,-100)],\
	[\
	(-2,-2, 100),(-1,-2, 100),( 0,-2, 100),( 1,-2,  32),( 2,-2,-100),\
	(-2,-1, 100),(-1,-1, 100),( 0,-1,  92),( 1,-1, -78),( 2,-1,-100),\
	(-2, 0, 100),(-1, 0, 100),( 0, 0,   0),( 1, 0,-100),( 2, 0,-100),\
	(-2, 1, 100),(-1, 1,  78),( 0, 1, -92),( 1, 1,-100),( 2, 1,-100),\
	(-2, 2, 100),(-1, 2, -32),( 0, 2,-100),( 1, 2,-100),( 2, 2,-100)],\
	[\
	(-2,-2,-100),(-1,-2,-100),( 0,-2,   0),( 1,-2, 100),( 2,-2, 100),\
	(-2,-1,-100),(-1,-1,-100),( 0,-1,   0),( 1,-1, 100),( 2,-1, 100),\
	(-2, 0,-100),(-1, 0,-100),( 0, 0,   0),( 1, 0, 100),( 2, 0, 100),\
	(-2, 1,-100),(-1, 1,-100),( 0, 1,   0),( 1, 1, 100),( 2, 1, 100),\
	(-2, 2,-100),(-1, 2,-100),( 0, 2,   0),( 1, 2, 100),( 2, 2, 100)],\
	[\
	(-2,-2,-100),(-1,-2,  32),( 0,-2, 100),( 1,-2, 100),( 2,-2, 100),\
	(-2,-1,-100),(-1,-1, -78),( 0,-1,  92),( 1,-1, 100),( 2,-1, 100),\
	(-2, 0,-100),(-1, 0,-100),( 0, 0,   0),( 1, 0, 100),( 2, 0, 100),\
	(-2, 1,-100),(-1, 1,-100),( 0, 1, -92),( 1, 1,  78),( 2, 1, 100),\
	(-2, 2,-100),(-1, 2,-100),( 0, 2,-100),( 1, 2, -32),( 2, 2, 100)],\
	[\
	(-2,-2, 100),(-1,-2, 100),( 0,-2, 100),( 1,-2, 100),( 2,-2, 100),\
	(-2,-1, -32),(-1,-1,  78),( 0,-1, 100),( 1,-1, 100),( 2,-1, 100),\
	(-2, 0,-100),(-1, 0, -92),( 0, 0,   0),( 1, 0,  92),( 2, 0, 100),\
	(-2, 1,-100),(-1, 1,-100),( 0, 1,-100),( 1, 1, -78),( 2, 1,  32),\
	(-2, 2,-100),(-1, 2,-100),( 0, 2,-100),( 1, 2,-100),( 2, 2,-100)]\
]
\end{lstlisting}

\begin{figure}[H]
\centering
\includegraphics[scale=0.5]{images/nevatia_babu-lena.bmp}
\caption{Nevatia-Babu $5x5$ operator with threshold = 12500}
\end{figure}

\section{How to Use}

There's only one program \texttt{edge\_detection.py}, to use it, just type command "\texttt{./edge\_detection.py --operator=[operator] [input file]}".

\end{document}
