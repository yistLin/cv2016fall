\documentclass[14pt,a4paper]{extarticle}

\usepackage[margin=2.54cm]{geometry}
\usepackage[slantfont,boldfont]{xeCJK}
\usepackage{listings}
\usepackage{float}
\usepackage{caption}
\usepackage{lastpage}
\usepackage{graphicx}
\usepackage{amsmath}
\graphicspath{ {../} }

\usepackage{fancyhdr}
\pagestyle{fancy}
\fancyhf{}
\lhead{\small{\jobname.pdf}}
\rhead{\small{Computer Vision 2016 Fall}}
\cfoot{\small{\thepage\ of \pageref{LastPage}}}

\usepackage{color}
\usepackage{listings}
\definecolor{codeyellow}{rgb}{0.6,0.4,0.1}
\definecolor{codeblue}{rgb}{0.1,0.2,0.6}
\definecolor{darkgray}{rgb}{0.3,0.3,0.3}
\lstset{
	language=Python,
	basicstyle=\footnotesize,
	keywordstyle=\color{codeblue},
	commentstyle=\color{darkgray},
	stringstyle=\color{codeyellow},
	numbers=left,
	numberstyle=\footnotesize,
	stepnumber=1,
	numbersep=5pt,
	backgroundcolor=\color{white},
	showspaces=false,
	showstringspaces=false,
	showtabs=false,
	frame=single,
	tabsize=2,
	captionpos=b,
	breaklines=true,
	breakatwhitespace=false,
	escapeinside={\%*}{*)}
}

\setCJKmainfont{PingFang TC}

\title{Computer Vision\\Homework 7 Report}
\author{林義聖\\B03902048}
\date{\today}

\begin{document}

\maketitle
\thispagestyle{fancy}

\section*{Introduction}

In this homework assignment, we're going to do \textit{thinning} on image. I use \textit{Python} as my programming language and \textit{Pillow} as my image library. It is a fork of \textit{PIL}, which is the original image library of \textit{Python}. And I use \textit{Pillow} for reading input image, and transfering image data into \textit{List} of \textit{Python}.

\begin{figure}[H]
\centering
\includegraphics[scale=0.5]{lena.bmp}
\caption{original lena.bmp}
\label{fig:lena.bmp}
\end{figure}

\section*{Program Structure}

There's only one program \textit{thinning.py} in my submission. And its structure is as following:
\begin{enumerate}
	\item Import required library (i.g PIL)
	\item Open input image
	\item Binarize the image
	\item Downsample the image
	\item Thin the image
		\begin{enumerate}
			\item Mark the border and interior pixels
			\item Mark deletable border pixels
			\item Calculate Yokoi connectivity number(YCN) of deletable pixels
			\item Remove deletable pixels which have $\texttt{YCN} = 1$
			\item Go back to \texttt{(a)} until no border pixels can be shrinked anymore
		\end{enumerate}
	\item Output thinned result to a text file
\end{enumerate}

\begin{lstlisting}
#!/usr/local/bin/python3.5
import sys
from PIL import Image

def binarize(data):
	...

def downsampling(data, hei, wid):
	...

def expand(data, hei, wid):
	...

def markIB(data, hei, wid):
	...

def mark_deletable(data, hei, wid):
	...

def h(b, c, d, e):
	...

def thinning(data, hei, wid):
	...

def main():
	# initial setup, handle system parameters
	...

	# get input image
	...

	# get 1D image data
	pixellist = list(img.getdata())

	# binarize the image
	data = binarize(pixellist)

	# downsampling the image
	data, hei, wid = downsampling(data, hei, wid)

	# thinning the image
	data, hei, wid = thinning(data, hei, wid)

	# output to text file
	...

if __name__ == '__main__':
	main()
\end{lstlisting}

\section*{Thinning}

To do thinning on an image, firstly, I need to mark the interior and border pixels on image.
\begin{lstlisting}
def markIB(data, hei, wid):
	ret = [0] * len(data)
	for y in range(1, hei-1):
		for x in range(1, wid-1):
			curr = y * wid + x
			count = 0
			if data[curr]:
				count += data[curr-1] + data[curr+1]
				count += data[curr-wid-1] + data[curr-wid] + data[curr-wid+1]
				count += data[curr+wid-1] + data[curr+wid] + data[curr+wid+1]
				if count < 8:
					ret[curr] = 1
				elif count == 8:
					ret[curr] = 2
	return ret
\end{lstlisting}

And then from marked image, I find the border pixel that next to some interior pixel and give them a specific label.
\begin{lstlisting}
def mark_deletable(data, hei, wid):
	for y in range(1, hei-1):
		for x in range(1, wid-1):
			curr = y * wid + x
			if data[curr] == 1:
				if 2 in [data[curr-1], data[curr+1], data[curr-wid-1], data[curr-wid], data[curr-wid+1], data[curr+wid-1], data[curr+wid], data[curr+wid+1]]:
					data[curr] = 3
\end{lstlisting}

In order to calculate Yokoi connectivity number, I define a function to tell the pattern of the neighbor of a pixel.
\begin{align*}
a_1 &= h(x_0, x_1, x_6, x_2)\\
a_2 &= h(x_0, x_2, x_7, x_3)\\
a_3 &= h(x_0, x_3, x_8, x_4)\\
a_4 &= h(x_0, x_4, x_5, x_1)
\end{align*}
And the deletable pixel $x$ that can really be shrinked is
\[ f(a_1, a_2, a_3, a_4, x) = g \text{ if excatly one of } a_1,a_2,a_3,a_4=1 \]

\begin{lstlisting}
def h(b, c, d, e):
	return 1 if b == c and (d != b or e != b) else 0
\end{lstlisting}

And finally, this is the complete steps to do thinning on a given image.
\begin{lstlisting}
def thinning(data, hei, wid):
	# expand the border of image
	exp_data, exp_hei, exp_wid = expand(data, hei, wid)

	prev_data = exp_data[:]

	while True:
		# mark border(1) and interior(2)
		marked = markIB(exp_data, exp_hei, exp_wid)

		# find deletable border(3)
		mark_deletable(marked, exp_hei, exp_wid)

		for y in range(1, exp_hei-1):
			for x in range(1, exp_wid-1):
				curr = y * exp_wid + x
				if marked[curr] == 3:
					# calculate yokoi connectivity number
					f = 0
					f += h(exp_data[curr], exp_data[curr+1], exp_data[curr-exp_wid+1], exp_data[curr-exp_wid])
					f += h(exp_data[curr], exp_data[curr-exp_wid], exp_data[curr-exp_wid-1], exp_data[curr-1])
					f += h(exp_data[curr], exp_data[curr-1], exp_data[curr+exp_wid-1], exp_data[curr+exp_wid])
					f += h(exp_data[curr], exp_data[curr+exp_wid], exp_data[curr+exp_wid+1], exp_data[curr+1])
					if f == 1:
						marked[curr] = 0
						exp_data[curr] = 0

		# compare shrinked data to previous data
		for i in range(len(exp_data)):
			if exp_data[i] != prev_data[i]:
				break
		else:
			break

	# backup previous image
	prev_data = exp_data[:]

	return exp_data, exp_hei, exp_wid
\end{lstlisting}

\section*{Result}

I use \texttt{*} to represent the white pixels that remain after thinning.
\lstset{
	language=,
	basicstyle=\tiny,
	backgroundcolor=\color{white},
	numbers=none,
	showspaces=false,
	showstringspaces=false,
	showtabs=false,
	basewidth={1.1em},
	frame=single,
	tabsize=2,
	captionpos=b,
	breaklines=true,
	breakatwhitespace=false,
	escapeinside={\%*}{*)}
}
\lstinputlisting{../thinned.txt}

\begin{figure}[H]
\centering
\includegraphics[scale=4]{thinned-lena.bmp}
\caption{thinned lena.bmp}
\label{fig:thinned-lena.bmp}
\end{figure}

\section*{How to Use}

There's only one executable program in my submission, and its name is \textit{thinning.py}. To run this program, just type this command ''\texttt{./thinning.py [input image] [output file]}''.

\end{document}
