\documentclass[14pt,a4paper]{extarticle}

\usepackage[margin=2.54cm]{geometry}
\usepackage[slantfont,boldfont]{xeCJK}
\usepackage{listings}
\usepackage{float}
\usepackage{caption}
\usepackage{lastpage}
\usepackage{graphicx}
\graphicspath{ {../images/} }

\usepackage{fancyhdr}
\pagestyle{fancy}
\fancyhf{}
\lhead{\small{\jobname.pdf}}
\rhead{\small{Computer Vision 2016 Fall}}
\cfoot{\small{\thepage\ of \pageref{LastPage}}}

\usepackage{color}
\usepackage{listings}
\definecolor{codeyellow}{rgb}{0.6,0.4,0.1}
\definecolor{codeblue}{rgb}{0.1,0.2,0.6}
\definecolor{darkgray}{rgb}{0.3,0.3,0.3}
\lstset{
	language=Python,
	basicstyle=\footnotesize,
	keywordstyle=\color{codeblue},
	commentstyle=\color{darkgray},
	stringstyle=\color{codeyellow},
	numbers=left,
	numberstyle=\footnotesize,
	stepnumber=1,
	numbersep=5pt,
	backgroundcolor=\color{white},
	showspaces=false,
	showstringspaces=false,
	showtabs=false,
	frame=single,
	tabsize=2,
	captionpos=b,
	breaklines=true,
	breakatwhitespace=false,
	escapeinside={\%*}{*)}
}

\setCJKmainfont{PingFang TC}

\title{Computer Vision\\Homework 6 Report}
\author{林義聖\\B03902048}
\date{\today}

\begin{document}

\maketitle
\thispagestyle{fancy}

\section*{Introduction}

In this homework assignment, we're going to calculate \textit{Yokoi Connectivity Number}.

I use \textit{Python} as my programming language and \textit{Pillow} as my image library. It is a fork of \textit{PIL}, which is the original image library of \textit{Python}. And I use \textit{Pillow} for reading input image, and transfering image data into \textit{List} of \textit{Python}.

\begin{figure}[H]
\centering
\includegraphics[scale=0.5]{lena.bmp}
\caption{original lena.bmp}
\label{fig:lena.bmp}
\end{figure}

\section*{Program Structure}

There's only one program in my submission, and I use program parameters to decide which algorithm we want to apply on the input image. All the things done in my program is:
\begin{enumerate}
	\item Import required library
	\item Open input image
	\item Binarize the image
	\item Downsample the image
	\item Calculate Yokoi connectivity number
	\item Output matrix to text file
\end{enumerate}

\begin{lstlisting}
import sys
from PIL import Image

# Setup system parameters
...

def binarize(data):
	...

def downsampling(data, hei, wid):
	...

def get_pattern(a1, a2, a3):
	...

def yokoi(data, hei, wid):
	...

def main():
	# get input image
	...

	# get 1D image data
	data_seq = list(in_img.getdata())

	# binarize the image
	bin_data = binarize(data_seq)

	# downsampling the image
	down_data, newhei, newwid = downsampling(bin_data, height, width)

	# yokoi neighborhood operate
	out_data = yokoi(down_data, newhei, newwid)

	# open output file and save
	...

if __name__ == '__main__':
    main()
\end{lstlisting}

\section*{Binarize}

We need to binarize the input image first. I binarize the image with threshold 127, and for my convenience, I simply binarize the image data to 1 or 0.
\begin{lstlisting}
def binarize(data):
	result = []
	for p in data:
		result.append(1 if p > 127 else 0)
	return result
\end{lstlisting}

\section*{Downsampling}

After binarizing the image, I downsample the image by choosing the topmost-left point of a $8\times8$ matrix.

\begin{lstlisting}
def downsampling(data, hei, wid):
	result = []
	offset = 8
	for y in range(0, hei, offset):
		for x in range(0, wid, offset):
			result.append(data[y * wid + x])
return (result, int(hei/offset), int(wid/offset))
\end{lstlisting}

\begin{figure}[H]
\centering
\includegraphics[scale=1]{downsampling.bmp}
\caption{downsampled lena.bmp}
\label{fig:downsampling.bmp}
\end{figure}

\section*{Yokoi Connectivity Number}

To calculate Yokoi connectivity number, I use the following steps:
\begin{enumerate}
	\item Expand the data matrix, append topmost, bottommost row and leftmost, rightmost column to it
	\item For pixels of original data matrix, get the pattern of its neighbors
	\item Calculate Yokoi connectivity number from patterns
	\item Keep doing step 2 and 3 to the end
\end{enumerate}

\begin{lstlisting}
def get_pattern(a1, a2, a3):
    if a1 == 1:
        if a2 == 1 and a3 == 1:
            return 'r'
        else:
            return 'q'
    else:
        return 's'

def yokoi(data, hei, wid):
	expd = [0] * (hei+2) * (wid+2)
	for y in range(hei):
		for x in range(wid):
			expd[(y+1) * (wid+2) + (x+1)] = data[y * wid + x]

	result = [' '] * (hei * wid)
	f = {'r':0, 'q':0, 's':0}
	for y in range(1, hei+1):
		for x in range(1, wid+1):
			f['r'] = 0
			f['q'] = 0
			f['s'] = 0
			if expd[y * (wid+2) + x] == 1:
				f[get_pattern(expd[y*(wid+2)+(x+1)],expd[(y-1)*(wid+2)+(x+1)],expd[(y-1)*(wid+2)+x])] += 1
				f[get_pattern(expd[(y-1)*(wid+2)+x],expd[(y-1)*(wid+2)+(x-1)],expd[y*(wid+2)+(x-1)])] += 1
				f[get_pattern(expd[y*(wid+2)+(x-1)],expd[(y+1)*(wid+2)+(x-1)],expd[(y+1)*(wid+2)+x])] += 1
				f[get_pattern(expd[(y+1)*(wid+2)+x],expd[(y+1)*(wid+2)+(x+1)],expd[y*(wid+2)+(x+1)])] += 1
			if f['r'] == 4:
				result[(y-1) * wid + (x-1)] = str(5)
			else:
				result[(y-1) * wid + (x-1)] = str(f['q'])
		else:
			result[(y-1) * wid + (x-1)] = ' '
	return result
\end{lstlisting}

\newpage

\section*{Result}

\lstset{
	language=,
	basicstyle=\tiny,
	backgroundcolor=\color{white},
	numbers=none,
	showspaces=false,
	showstringspaces=false,
	showtabs=false,
	basewidth={1.1em},
	frame=single,
	tabsize=2,
	captionpos=b,
	breaklines=true,
	breakatwhitespace=false,
	escapeinside={\%*}{*)}
}
\lstinputlisting{../yokoi.txt}

\section*{How to Use}

There's only one executable program in my submission, and its name is \textit{yokoi.py}. To run this program, just type this command ''\texttt{./yokoi.py [input image] [output file]}''.

\end{document}
